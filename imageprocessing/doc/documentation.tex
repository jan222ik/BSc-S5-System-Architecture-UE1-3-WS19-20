% Preamble
\documentclass[a4paper, 11pt]{scrreprt}
% Packages
\usepackage[utf8]{inputenc}
\usepackage[T1]{fontenc}
\usepackage{amssymb}
\usepackage{graphicx}
\usepackage{setspace}
\usepackage[english]{babel}
\usepackage{listings}
\usepackage{acronym}
\usepackage{eurosym}
\usepackage{newtxmath,newtxtext}
%\usepackage{minted} %Must be before csquotes
\usepackage{csquotes}
\usepackage{enumitem}
\usepackage{float}
\usepackage{tocloft}
\usepackage{url}
\usepackage{geometry}
\geometry{
a4paper,
total={170mm,257mm},
left=10mm,
top=10mm,
}

\setstretch{1.5}

\title{Exercise 2 — Image Processing Documentation}
\author{Matthias Rupp, Janik Mayr}
% Document
\begin{document}
\maketitle
placeholder text :D — interesting

% Please add the following required packages to your document preamble:
% \usepackage{graphicx}
\begin{table}[]
\resizebox{\textwidth}{!}{%
\begin{tabular}{|l|l|l|l|}
\hline
Name              & Necessary & Definition                                         & Default                                                        \\ \hline
-src              & yes       & Source image                                       & Path: imageprocessing/src/main/resources/loetstellen.jpg       \\ \hline
-acc              & yes       & Accuracy                                           & 3                                                              \\ \hline
-pull             & yes       & Switch to Pullpipe                                 & false                                                          \\ \hline
-exptCoords       & /         & File with expected Coords                          & Path: imageprocessing/src/main/resources/expectedCentroids.txt \\ \hline
-disksImgOut      & /         & Output for created DiskImg after Erosion           & Path: imageprocessing/target/disks.png                         \\ \hline
-diskReportOut    & /         & Output for created Report                          & Path: imageprocessing/target/report.txt                        \\ \hline
-shapeTypeErode   & /         & Shape type for erode filter Range{[}0 - 3{]}       & 0                                                              \\ \hline
-kernelSizeErode  & /         & Kernel size for erode filter                       & 2                                                              \\ \hline
-kernelSizeMedian & /         & Kernel size for median filter                      & 19                                                             \\ \hline
-removeFirstN     & /         & Removes the first n found disks in the given image & 1                                                              \\ \hline
\end{tabular}%
}
\end{table}
\end{document}
