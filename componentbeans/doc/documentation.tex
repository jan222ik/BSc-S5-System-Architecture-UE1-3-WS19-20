% Preamble
\documentclass[a4paper, 11pt]{scrreprt}
% Packages
\usepackage[utf8]{inputenc}
\usepackage[T1]{fontenc}
\usepackage{amssymb}
\usepackage{graphicx}
\usepackage{setspace}
\usepackage[english]{babel}
\usepackage{listings}
\usepackage{acronym}
\usepackage{eurosym}
\usepackage{newtxmath,newtxtext}
%\usepackage{minted} %Must be before csquotes
\usepackage{csquotes}
\usepackage{enumitem}
\usepackage{float}
\usepackage{tocloft}
\usepackage{url}
\usepackage{geometry}
\geometry{
a4paper,
total={170mm,257mm},
left=10mm,
top=10mm,
textwidth=190mm
}
\usepackage[style=authoryear,citestyle=authoryear,backend=bibtex]{biblatex}

\setstretch{1.5}

\title{Exercise 3 — Component Beans for Beanbox Documentation}
\author{Matthias Rupp, Janik Mayr}
% Document
\begin{document}
\maketitle

\chapter{Start & Bean Properties}
To start the program, simply drop the jar into the jars folder of the beanbox. Upon starting said beanbox, the Beans
should now be visible. Since the jar is a fat-jar, no libraries need to be added.
\begin{table}[H]
\resizebox{\textwidth}{!}{%
\begin{tabular}{|l|l|l|}
\hline
BeanName & PropertyName & Definition & Default                                                        \\ \hline
ErodeFilterBean  & shapeType & Shape used for Erode Filter, Range  & 0      \\ \hline
ErodeFilterBean  & kernalSize &  KernelSize used for the Erode Filter & 2                                                              \\ \hline
FindRadiusFilter  & threshold & Threshold for radius finding & 100                                                          \\ \hline
ImgSource & imgPath & Path from where the source is supposed to load the image & /) \\ \hline
MedianFilter & kSize & kernelSize for MedianFilter & 19                       \\ \hline
QualityCheckFilter & expectedCoordinatesFile & Path to the file with the expected coordinates &  /                     \\ \hline
QualityCheckFilter & accuracy & Accuracy for comparison between actual and expected coordinates & 3.0                                                              \\ \hline
ROIFilter & x & Distance from starting coordinate (0|0) to start of ROI in x-direction & 0                                                              \\ \hline
ROIFilter & y & Distance from starting coordinate (0|0) to start of ROI in y-direction & 35                                                             \\ \hline
ROIFilter & width & Width of the ROI-Rectangle & 448                                                             \\ \hline
ROIFilter & height & Height of the ROI-Rectangle & 80                                                             \\ \hline
SinkImpl & outputPath & Path to where the Sink writes the Report & /                                                            \\ \hline
SinkImpl & accuracy & Accuracy that is written into the report (should match QualityCheckFilter) & 3.0                                                             \\ \hline
ThresholdFilter & thresh & Threshold for filtering & 30                                                          \\ \hline
ThresholdFilter & type & Type of threshold-filter used & 1                                                           \\ \hline
\end{tabular}%
}
\caption{Table of command-line arguments}
\label{tab:args-table}
\end{table}
\chapter{In-Depth Description}


\printbibliography
\end{document}
